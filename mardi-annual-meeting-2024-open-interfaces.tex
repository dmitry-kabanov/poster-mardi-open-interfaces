\documentclass{mmposter}

\colorlet{CEmphasis1}{clustercassis}
\colorlet{CEmphasis2}{clusterblue}

\usepackage{lipsum}
\usepackage{blindtext}
\usepackage{tikz}
\usepackage{amsmath}
\usepackage{derivative}
\usepackage{hyperref}
\hypersetup{
	colorlinks,
	citecolor=black,
	urlcolor=CEmphasis1,
}
\usepackage{minted}

\title{Improving Interoperability\\in Scientific Computing\\via MaRDI Open Interfaces}
\authors{Contributors:
	Dmitry I.\ Kabanov, Stephan Rave, Mario Ohlberger
}
\renewcommand{\refname}{References}

\newcommand{\OIF}{\textsc{MaRDI Open Interfaces}\xspace}

\newcounter{hilfszaehler}

\graphicspath{{./assets/}}

\usepackage[natbib=true, style=authoryear, giveninits=true, uniquename=init]{biblatex}
% While editing, it is convenient to use an external BIB file,
% which is, nevertheless, is not commited to the Git repository.
% To extract only used citations and put them to `bibliography`,
% use this command in the terminal:
% jabref --nogui --aux main.aux,bibliography.bib bibliography-external.bib
\IfFileExists{./bibliography-external.bib}{%
	\addbibresource{bibliography-external.bib}
}{%
	\addbibresource{bibliography.bib}
}


\begin{document}

\maketitle

\section*{Summary}
Computational scientists often face two obstacles while conducting
numerical experiments.
First, there are several popular languages used by numerical
software packages as well as different preferences for a ``driving''
language, in which an experiment's logic is formulated.
Connecting different languages and packages directly requires
developing bindings to for these packages, which can amount
to a significant work.
Second, the programming interfaces between even packages
that solve the same numerical problem, have different interfaces.
Therefore, it requires a significant effort from a computational
scientist, if there is a need to switch from one numerical package
to another, in the flow of the computational project.

The goal of \OIF{} is to improve interoperability in Scientific
Computing by alleviating these obstacles.
Similar to software packages such as \textsc{PyMOR}~\citep{Milk2016},
\textsc{SUNDIALS}~\citep{GardnerEtAl2022},
or \textsc{PreCICE}~\citep{Chourdakis2022},
we aim to abstract out discrepancies between
different solvers, so that computational scientists would spend less time
on connecting them together.

Our \textbf{objectives} during the project are:
\begin{itemize}
  \item Develop library that passes data between languages automatically
  \item Develop generic interfaces for typical numerical problems,
  such as optimization or integration of differential equations
  \item Spread information about these interfaces to encourage
  the scientific-computing community to program against these interfaces.
\end{itemize}

\section*{Software Architecture and Data Flow}
\includegraphics[width=\columnwidth]{arch}

\textbf{User} requests an implementation of an Open Interface
and interacts with the implementation only via a \textbf{Gateway},
so that the discrepancies between different implementations in terms
of functions signatures and order of invocation become transparent
to the \textbf{User}.

The function arguments are converted from the user's language
to a C intermediate representation inside a \textbf{Converter}
that passes them further.
The \textbf{Dispatch} is responsible for loading an implementation
and its corresponding runtime (\textbf{Bridge}).
The \textbf{Bridge} converts data from the intermediate representation
to the native data types of the implementation and invokes
the requested function, which also occurs via \textbf{Interface},
that is, the implementation is invoked via an adapter.

\newpage

\section*{Example usage}
Inviscid Burgers' equation:
\begin{align*}
  &\pdv{u}{t} + \pdv{\left( u^{2} / 2 \right)}{x} = 0,
  \quad t \in [0, 2], \quad x \in [0, 2] \\
  &u(t, 0) = 0.5 - 0.25 \sin \left( \pi x \right)\\
  &u(t, 0) = u(t, 2)
\end{align*}
\begin{minipage}{\dimexpr0.46\columnwidth - 2\tabcolsep}
  \textbf{Implementations}:
  \begin{itemize}
    \item Sundials CVODE,\\Adams--Moulton method
    \item \texttt{scipy.integrate.ode},\\Dormand--Prince method
    \item \texttt{OrdinaryDiffEq},\\Tsitouras method
  \end{itemize}
\end{minipage}\hfill%
\begin{minipage}{\dimexpr0.54\columnwidth - 2\tabcolsep}
  % \centering
  \includegraphics[width=\linewidth]{ivp_c_burgers_eq}
\end{minipage}

\vspace{2em}
\textbf{User code in Python:}
\begin{minted}[autogobble, beameroverlays, escapeinside=||]{Python}
  from oif.interfaces.ivp import IVP
  ...

  # `BurgersEquationProblem` is a utility user class.
  impl = "scipy_ode"
  problem = BurgersEquationProblem(N=1001)
  s = IVP(impl)
  s.set_initial_value(problem.y0, problem.t0)
  s.set_rhs_fn(problem.compute_rhs)

  times = np.linspace(problem.t0, problem.tfinal, num=11)

  soln = [y0]
  for t in times[1:]:
      s.integrate(t)
      soln.append(s.y)
\end{minted}


\section*{Conclusions}
We demonstrated how \OIF{} can be used to improve
interoperability in Scientific Computing.
Using this software package, computational scientists can connect numerical
solvers written in different languages, without explicitly writing bindings
to these solvers.
Besides, the library allows switching between different implementations
of a numerical problem without extensive modifications of a user's code,
that facilitates benchmarking numerical software in terms of performance
and obtained results.


% \section*{Connections to other MaRDI subprojects}

% \begin{itemize}[align=left]
%   \item[\color{CEmphasis1}M1.4:] ``Predefined Software Environments'' subproject
%         helps scientists with reproducibility by making it easier to have
%         a set of installed numerical packages.
%   \item[\color{CEmphasis1}M2.3:] ``Benchmark Framework'' subproject
%         compares numerical software packages, which can be facilitated
%         using \OIF{}.
% \end{itemize}

\section*{Acknowledgments}
The authors would like to thank for the provided funding
the National Research Data Infrastructure,
project number~460135501, NFDI~29/1 “MaRDI – Mathematical
Research Data Initiative”
and
the German Research Foundation,
Germany's Excellence Strategy EXC~2044-390685587,
``Mathematics Münster: Dynamics--Geometry--Structure''.

\printbibliography{}

% \begin{thebibliography}{1}
%   \setlength{\itemsep}{1pt}
%   \setlength{\parskip}{1.5pt}

%   \scriptsize{

%   \bibitem[1]{PyMOR}
%   Ren{\'{e}}, M., Rave, S., and Schindler, F.
%   \newblock pyMOR -- Generic Algorithms and Interfaces for Model Order Reduction, 2016.
%   \newblock doi:10.1137/15m1026614.
%   }
% \end{thebibliography}

\end{document}
